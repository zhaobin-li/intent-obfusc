\begin{table*}
\centering
\begin{threeparttable}
\begin{tabular}[t]{llllll}
\toprule
\multicolumn{3}{c}{ } & \multicolumn{3}{c}{Attack Losses\textsuperscript{c}} \\
\cmidrule(l{3pt}r{3pt}){4-6}
\multicolumn{3}{c}{ } & \multicolumn{2}{c}{Targeted} & \multicolumn{1}{c}{ } \\
\cmidrule(l{3pt}r{3pt}){4-5}
Detectors & Stages\textsuperscript{a} & COCO mAP\textsuperscript{b} & Vanishing & Mislabeling & Untargeted\textsuperscript{d}\\
\midrule
\cellcolor{gray!10}{YOLOv3} & \cellcolor{gray!10}{1} & \cellcolor{gray!10}{33.7} & \cellcolor{gray!10}{Object} & \cellcolor{gray!10}{Class} & \cellcolor{gray!10}{Class, Box, Object}\\
SSD & 1 & 29.5 & Class & Class & Class, Box\\
\cellcolor{gray!10}{RetinaNet} & \cellcolor{gray!10}{1} & \cellcolor{gray!10}{36.5} & \cellcolor{gray!10}{Class} & \cellcolor{gray!10}{Class} & \cellcolor{gray!10}{Class, Box}\\
\makecell[l]{Faster\\R-CNN} & 2 & 37.4 & \makecell[l]{RPN: Object;\\Det: Class} & Det: Class & \makecell[l]{RPN: Object, Box;\\Det: Class, Box}\\
\cellcolor{gray!10}{\makecell[l]{Cascade\\R-CNN}} & \cellcolor{gray!10}{2} & \cellcolor{gray!10}{40.3} & \cellcolor{gray!10}{\makecell[l]{RPN 1: Object;\\RPNs 2, 3 + Det: Class}} & \cellcolor{gray!10}{\makecell[l]{RPNs 2, 3: Class;\\Det: Class}} & \cellcolor{gray!10}{\makecell[l]{RPN 1: Object, Box;\\RPNs 2, 3 + Det: Class, Box}}\\
\bottomrule
\end{tabular}
\begin{tablenotes}
\item[a] In general, 1-stage detectors are quicker whereas 2-stage detectors are more accurate, though the 1-stage RetinaNet aims to be both quick and accurate. In a 2-stage detector, the input image passes through a Region Proposal Network (RPN) stage and a detection (Det) stage.
\item[b] COCO mean Average Precision (mAP) is the primary metric on the COCO challenge.
\item[c] The training losses in detectors typically include the box regression loss (Box), the class loss on the 80 COCO labels and/or the background class (Class), and the objectness loss on categorizing an image region as background or object (Object).
\item[d] Untargeted attack targets all training losses in a model, i.e. the backpropagation loss.
\end{tablenotes}
\end{threeparttable}
\centering
\caption{\label{tab:models_table}Detection models and attack losses. Full details are given in Appendix \ref{app:mod_los}.}
\end{table*}
